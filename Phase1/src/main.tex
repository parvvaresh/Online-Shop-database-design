\documentclass[12pt]{article}
\usepackage{amsthm,amssymb,amsmath,amsfonts}
\usepackage[a4paper, top=25mm, bottom=30mm, left=25mm, right=25mm]{geometry}
\usepackage[pagebackref=false,colorlinks,linkcolor=black,citecolor=black]{hyperref}
\usepackage[nameinlink]{cleveref}
 \AtBeginDocument{%
    \crefname{equation}{برابری}{equations}%
    \crefname{chapter}{فصل}{chapters}%
    \crefname{section}{بخش}{sections}%
    \crefname{appendix}{پیوست}{appendices}%
    \crefname{enumi}{مورد}{items}%
    \crefname{footnote}{زیرنویس}{footnotes}%
    \crefname{figure}{شکل}{figures}%
    \crefname{table}{جدول}{tables}%
    \crefname{theorem}{قضیه}{theorems}%
    \crefname{lemma}{لم}{lemmas}%
    \crefname{corollary}{نتیجه}{corollaries}%
    \crefname{proposition}{گزاره}{propositions}%
    \crefname{definition}{تعریف}{definitions}%
    \crefname{result}{نتیجه}{results}%
    \crefname{example}{مثال}{examples}%
    \crefname{remark}{نکته}{remarks}%
    \crefname{note}{یادداشت}{notes}%
    \crefname{observation}{مشاهده}{observations}%
    \crefname{algorithm}{الگوریتم}{algorithms}%
    \crefname{cproof}{برهان}{cproofs}%
}

\usepackage{tikz}
\usepackage{graphicx}
\usepackage{color}

\usepackage{setspace}
\doublespacing

\usepackage{titletoc}
\usepackage{tocloft}
\usepackage{enumitem}

\usepackage{algorithm}
% \usepackage[noend]{algpseudocode}
\usepackage[noend]{algorithmic}
\renewcommand{\algorithmicrequire}{\textbf{Input:}}
\renewcommand{\algorithmicensure}{\textbf{Output:}}

\usepackage{tabularx}
\makeatletter
\newcommand{\multiline}[1]{%
  \begin{tabularx}{\dimexpr\linewidth-\ALG@thistlm}[t]{@{}X@{}}
    #1
  \end{tabularx}
}
\makeatother

\usepackage{float}
\usepackage{verbatim}
\makeindex
\usepackage{sectsty}
\usepackage{xepersian}
\SepMark{-}
\settextfont[Scale=1.2,Path=fonts/,BoldFont=B Nazanin Bold.ttf]{B Nazanin.ttf}
\setlatintextfont{Times New Roman}
\renewcommand{\labelitemi}{$\bullet$}

\theoremstyle{definition}
\newtheorem{definition}{تعریف}[section]
\newtheorem{remark}[definition]{نکته}
\newtheorem{note}[definition]{یادداشت}
\newtheorem{example}[definition]{نمونه}
\newtheorem{question}[definition]{سوال}
\newtheorem{remember}[definition]{یاداوری}
\newtheorem{observation}[definition]{مشاهده}
\theoremstyle{theorem}
\newtheorem{theorem}[definition]{قضیه}
\newtheorem{lemma}[definition]{لم}
\newtheorem{proposition}[definition]{گزاره}
\newtheorem{corollary}[definition]{نتیجه}
\newtheorem*{cproof}{برهان}



\begin{document}
\fontsize{12pt}{14pt}\selectfont

\begin{minipage}{0.1\textwidth}
\includegraphics[width=2cm]{etc/aut}
\end{minipage}%
\hfill%
\begin{minipage}{0.6\textwidth}\centering
\fontsize{10pt}{10pt}\selectfont
پروژه درس پایگاه داده \\
فروشگاه آنلاین\\ 
\vspace{0.25cm}
\begingroup
\fontsize{8pt}{8pt}\selectfont
دانشگاه صنعتی امیرکبیر، دانشکده ریاضی و علوم کامپیوتر \\
اردیبهشت 1403  \\
\endgroup
\end{minipage}%
\hfill%
\begin{minipage}{0.1\textwidth}
\includegraphics[width=3cm]{etc/database log.jpg}
\end{minipage}

\vspace{0.5cm}

\noindent\rule{\textwidth}{1pt}

% \maketitle

\renewcommand{\abstractname}{طراحی پایگاه داده فروشگاه آنلاین}
\begin{abstract}
این پروژه شامل طراحی و پیاده‌سازی یک پایگاه داده برای یک فروشگاه آنلاین است. هدف این پروژه مدیریت کاربران، مدیران، محصولات، دسته‌بندی‌ها، برندها، سفارشات، سبد خرید، تاریخچه خرید، نظرات، اطلاعات حمل و نقل و تخفیفات است. ویژگی‌های این پایگاه داده به شرح زیر است:
\end{abstract}
\begin{enumerate}
    \item \textbf{مدیریت کاربران}:
    \begin{itemize}
        \item ثبت‌نام کاربران جدید با اطلاعات کامل.
        \item امکان ورود کاربران به سیستم با استفاده از نام کاربری و رمز عبور.
        \item به‌روزرسانی اطلاعات پروفایل کاربران شامل نام، ایمیل، شماره تماس و آدرس.
    \end{itemize}
    \vspace{0.3cm}
    
    \item \textbf{مدیریت مدیران}:
    \begin{itemize}
        \item ثبت‌نام مدیران جدید و ذخیره اطلاعات آن‌ها.
        \item ورود مدیران به سیستم با استفاده از نام کاربری و رمز عبور.
        \item به‌روزرسانی اطلاعات پروفایل مدیران شامل نام کاربری و ایمیل.
    \end{itemize}
    \vspace{0.3cm}

    \item \textbf{مدیریت محصولات}:
    \begin{itemize}
        \item اضافه کردن محصولات جدید با مشخصات کامل.
        \item ویرایش و به‌روزرسانی اطلاعات محصولات موجود.
        \item حذف محصولات قدیمی و ناموجود.
    \end{itemize}
    \vspace{0.3cm}
    
    \item \textbf{مدیریت دسته‌بندی‌ها}:
    \begin{itemize}
        \item اضافه کردن دسته‌بندی‌های جدید برای محصولات.
        \item ویرایش و به‌روزرسانی دسته‌بندی‌های موجود.
        \item حذف دسته‌بندی‌های غیر ضروری.
    \end{itemize}
    \vspace{0.3cm}
    
    \item \textbf{مدیریت برندها}:
    \begin{itemize}
        \item اضافه کردن برندهای جدید برای محصولات.
        \item ویرایش و به‌روزرسانی اطلاعات برندهای موجود.
        \item حذف برندهای قدیمی و غیر فعال.
    \end{itemize}
    \vspace{0.3cm}
    
    \item \textbf{مدیریت سفارشات کاربران}:
    \begin{itemize}
        \item ثبت سفارشات جدید توسط کاربران.
        \item پیگیری وضعیت سفارشات از ثبت تا تحویل.
        \item مشاهده جزئیات هر سفارش توسط کاربران و مدیران.
    \end{itemize}
    \vspace{0.3cm}
    
    \item \textbf{جزئیات سفارشات}:
    \begin{itemize}
        \item ذخیره اطلاعات دقیق هر سفارش شامل محصولات سفارش داده شده، تعداد و قیمت.
        \item مشاهده و مدیریت جزئیات سفارشات توسط کاربران و مدیران.
    \end{itemize}
    \vspace{0.3cm}
    
    \item \textbf{مدیریت سبد خرید}:
    \begin{itemize}
        \item ایجاد و به‌روزرسانی سبد خرید برای هر کاربر.
        \item اضافه کردن و حذف آیتم‌های موجود در سبد خرید.
        \item محاسبه مبلغ کل سبد خرید به صورت خودکار.
    \end{itemize}
    \vspace{0.3cm}
    
    \item \textbf{تاریخچه خرید}:
    \begin{itemize}
        \item ذخیره و مشاهده تاریخچه خریدهای انجام شده توسط کاربران.
        \item ارائه گزارش‌های مربوط به خریدهای گذشته به کاربران و مدیران.
    \end{itemize}
    \vspace{0.3cm}
    
    \item \textbf{مدیریت نظرات}:
    \begin{itemize}
        \item ثبت نظرات کاربران درباره محصولات.
        \item مشاهده نظرات توسط سایر کاربران و مدیران.
        \item مدیریت و حذف نظرات نامناسب توسط مدیران.
    \end{itemize}
    \vspace{0.3cm}
    
    \item \textbf{اطلاعات حمل و نقل}:
    \begin{itemize}
        \item ثبت اطلاعات حمل و نقل شامل شماره پیگیری، حامل، تاریخ ارسال و تحویل.
        \item پیگیری وضعیت حمل و نقل سفارشات توسط کاربران و مدیران.
    \end{itemize}
    \vspace{0.3cm}
    
    \item \textbf{مدیریت تخفیفات}:
    \begin{itemize}
        \item اضافه کردن تخفیفات و پیشنهادات ویژه جدید.
        \item ویرایش و به‌روزرسانی تخفیفات موجود.
        \item اعمال تخفیفات در زمان مناسب برای محصولات مشخص.
    \end{itemize}
    \vspace{0.3cm}
    

    

\end{enumerate}

\section{فاز اول: طراحی نمودار \(ERD\) - مهلت تحویل: 4 خرداد}
در این فاز، باید یک نمودار \(ERD (Entity-Relationship Diagram) \) برای پایگاه داده فروشگاه آنلاین تهیه شود. این نمودار باید شامل تمامی جداول مورد نیاز، روابط بین آن‌ها و ویژگی‌های مربوطه باشد.به طوری که تمام نیاز های مورد نیاز که در بالا گفتیم برآورده شود. 




\textbf{دقت کنید روابط چند به چند مجاز نیست.}

در صورت هر گونه ابهام و سوال به این ایدی ها پیام دهید :‌

\href{https://t.me/Ali_Abdollahian_Noghabi}{پرورش}

\href{https://t.me/parvvaresh}{‫ﻋﺒﺪﺍﻟﻬﻴﺎﻥ‬}



\end{document}
