\documentclass[12pt]{article}
\usepackage{amsthm,amssymb,amsmath,amsfonts}
\usepackage[a4paper, top=25mm, bottom=30mm, left=25mm, right=25mm]{geometry}
\usepackage[pagebackref=false,colorlinks,linkcolor=black,citecolor=black]{hyperref}
\usepackage[nameinlink]{cleveref}
 \AtBeginDocument{%
    \crefname{equation}{برابری}{equations}%
    \crefname{chapter}{فصل}{chapters}%
    \crefname{section}{بخش}{sections}%
    \crefname{appendix}{پیوست}{appendices}%
    \crefname{enumi}{مورد}{items}%
    \crefname{footnote}{زیرنویس}{footnotes}%
    \crefname{figure}{شکل}{figures}%
    \crefname{table}{جدول}{tables}%
    \crefname{theorem}{قضیه}{theorems}%
    \crefname{lemma}{لم}{lemmas}%
    \crefname{corollary}{نتیجه}{corollaries}%
    \crefname{proposition}{گزاره}{propositions}%
    \crefname{definition}{تعریف}{definitions}%
    \crefname{result}{نتیجه}{results}%
    \crefname{example}{مثال}{examples}%
    \crefname{remark}{نکته}{remarks}%
    \crefname{note}{یادداشت}{notes}%
    \crefname{observation}{مشاهده}{observations}%
    \crefname{algorithm}{الگوریتم}{algorithms}%
    \crefname{cproof}{برهان}{cproofs}%
}

\usepackage{tikz}
\usepackage{graphicx}
\usepackage{color}

\usepackage{setspace}
\doublespacing

\usepackage{titletoc}
\usepackage{tocloft}
\usepackage{enumitem}

\usepackage{algorithm}
% \usepackage[noend]{algpseudocode}
\usepackage[noend]{algorithmic}
\renewcommand{\algorithmicrequire}{\textbf{Input:}}
\renewcommand{\algorithmicensure}{\textbf{Output:}}

\usepackage{tabularx}
\makeatletter
\newcommand{\multiline}[1]{%
  \begin{tabularx}{\dimexpr\linewidth-\ALG@thistlm}[t]{@{}X@{}}
    #1
  \end{tabularx}
}
\makeatother

\usepackage{float}
\usepackage{verbatim}
\makeindex
\usepackage{sectsty}
\usepackage{xepersian}
\SepMark{-}
\settextfont[Scale=1.2,Path=fonts/,BoldFont=B Nazanin Bold.ttf]{B Nazanin.ttf}
\setlatintextfont{Times New Roman}
\renewcommand{\labelitemi}{$\bullet$}

\theoremstyle{definition}
\newtheorem{definition}{تعریف}[section]
\newtheorem{remark}[definition]{نکته}
\newtheorem{note}[definition]{یادداشت}
\newtheorem{example}[definition]{نمونه}
\newtheorem{question}[definition]{سوال}
\newtheorem{remember}[definition]{یاداوری}
\newtheorem{observation}[definition]{مشاهده}
\theoremstyle{theorem}
\newtheorem{theorem}[definition]{قضیه}
\newtheorem{lemma}[definition]{لم}
\newtheorem{proposition}[definition]{گزاره}
\newtheorem{corollary}[definition]{نتیجه}
\newtheorem*{cproof}{برهان}



\begin{document}
\fontsize{12pt}{14pt}\selectfont

\begin{minipage}{0.1\textwidth}
\includegraphics[width=2cm]{etc/aut}
\end{minipage}%
\hfill%
\begin{minipage}{0.6\textwidth}\centering
\fontsize{10pt}{10pt}\selectfont
پروژه درس پایگاه داده \\
فروشگاه آنلاین\\ 
\vspace{0.25cm}
\begingroup
\fontsize{8pt}{8pt}\selectfont
دانشگاه صنعتی امیرکبیر، دانشکده ریاضی و علوم کامپیوتر \\
اردیبهشت 1403  \\
\endgroup
\end{minipage}%
\hfill%
\begin{minipage}{0.1\textwidth}
\includegraphics[width=3cm]{etc/database log.jpg}
\end{minipage}

\vspace{0.5cm}

\noindent\rule{\textwidth}{1pt}

% \maketitle

\renewcommand{\abstractname}{طراحی پایگاه داده فروشگاه آنلاین}


\section{فاز دوم: نرمال‌سازی  - مهلت تحویل: 22 خرداد}

در این فاز شما باید جداول خود را تا سطح سوم نرمال‌سازی کنید. پس از نرمال‌سازی، مجدداً نمودار \(ERD (Entity-Relationship Diagram)\) را رسم کنید.

در این فاز می‌بایست گزارشی از فرایند نرمال‌سازی خود بنویسید و مراحل نرمال‌سازی از سطح اول تا سطحی که جدول شما شرایط آن را دارا است را شرح دهید.

به عنوان مثال:
ابتدا جداول را از نظر شرایط نرمال فرم اول بررسی می‌کنیم. اگر شرایط را نقض نکرده باشد، یعنی جدول در نرمال فرم اول است. در غیر این صورت، جدول را به نرمال فرم اول تبدیل کرده و سپس بعد از بررسی دوباره شرایط نرمال فرم اول، به سراغ نرمال فرم دوم می‌رویم و مراحل قبل را تکرار می‌کنیم.

مراحل نرمال‌سازی را به دقت بررسی و مستندسازی کنید تا اطمینان حاصل شود که جداول شما تا سطح سوم نرمال‌سازی به درستی انجام شده‌اند.



در صورت هر گونه ابهام و سوال به این ایدی ها پیام دهید :‌

\href{https://t.me/Ali_Abdollahian_Noghabi}{پرورش}

\href{https://t.me/parvvaresh}{‫ﻋﺒﺪﺍﻟﻬﻴﺎﻥ‬}



\end{document}
