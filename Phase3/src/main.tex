\documentclass[12pt]{article}
\usepackage{amsthm,amssymb,amsmath,amsfonts}
\usepackage[a4paper, top=25mm, bottom=30mm, left=25mm, right=25mm]{geometry}
\usepackage[pagebackref=false,colorlinks,linkcolor=black,citecolor=black]{hyperref}
\usepackage[nameinlink]{cleveref}
 \AtBeginDocument{%
    \crefname{equation}{برابری}{equations}%
    \crefname{chapter}{فصل}{chapters}%
    \crefname{section}{بخش}{sections}%
    \crefname{appendix}{پیوست}{appendices}%
    \crefname{enumi}{مورد}{items}%
    \crefname{footnote}{زیرنویس}{footnotes}%
    \crefname{figure}{شکل}{figures}%
    \crefname{table}{جدول}{tables}%
    \crefname{theorem}{قضیه}{theorems}%
    \crefname{lemma}{لم}{lemmas}%
    \crefname{corollary}{نتیجه}{corollaries}%
    \crefname{proposition}{گزاره}{propositions}%
    \crefname{definition}{تعریف}{definitions}%
    \crefname{result}{نتیجه}{results}%
    \crefname{example}{مثال}{examples}%
    \crefname{remark}{نکته}{remarks}%
    \crefname{note}{یادداشت}{notes}%
    \crefname{observation}{مشاهده}{observations}%
    \crefname{algorithm}{الگوریتم}{algorithms}%
    \crefname{cproof}{برهان}{cproofs}%
}

\usepackage{tikz}
\usepackage{graphicx}
\usepackage{color}

\usepackage{setspace}
\doublespacing

\usepackage{titletoc}
\usepackage{tocloft}
\usepackage{enumitem}

\usepackage{algorithm}
% \usepackage[noend]{algpseudocode}
\usepackage[noend]{algorithmic}
\renewcommand{\algorithmicrequire}{\textbf{Input:}}
\renewcommand{\algorithmicensure}{\textbf{Output:}}

\usepackage{tabularx}
\makeatletter
\newcommand{\multiline}[1]{%
  \begin{tabularx}{\dimexpr\linewidth-\ALG@thistlm}[t]{@{}X@{}}
    #1
  \end{tabularx}
}
\makeatother

\usepackage{float}
\usepackage{verbatim}
\makeindex
\usepackage{sectsty}
\usepackage{xepersian}
\SepMark{-}
\settextfont[Scale=1.2,Path=fonts/,BoldFont=B Nazanin Bold.ttf]{B Nazanin.ttf}
\setlatintextfont{Times New Roman}
\renewcommand{\labelitemi}{$\bullet$}

\theoremstyle{definition}
\newtheorem{definition}{تعریف}[section]
\newtheorem{remark}[definition]{نکته}
\newtheorem{note}[definition]{یادداشت}
\newtheorem{example}[definition]{نمونه}
\newtheorem{question}[definition]{سوال}
\newtheorem{remember}[definition]{یاداوری}
\newtheorem{observation}[definition]{مشاهده}
\theoremstyle{theorem}
\newtheorem{theorem}[definition]{قضیه}
\newtheorem{lemma}[definition]{لم}
\newtheorem{proposition}[definition]{گزاره}
\newtheorem{corollary}[definition]{نتیجه}
\newtheorem*{cproof}{برهان}



\begin{document}
\fontsize{12pt}{14pt}\selectfont

\begin{minipage}{0.1\textwidth}
\includegraphics[width=2cm]{etc/aut}
\end{minipage}%
\hfill%
\begin{minipage}{0.6\textwidth}\centering
\fontsize{10pt}{10pt}\selectfont
پروژه درس پایگاه داده \\
فروشگاه آنلاین\\ 
\vspace{0.25cm}
\begingroup
\fontsize{8pt}{8pt}\selectfont
دانشگاه صنعتی امیرکبیر، دانشکده ریاضی و علوم کامپیوتر \\
اردیبهشت 1403  \\
\endgroup
\end{minipage}%
\hfill%
\begin{minipage}{0.1\textwidth}
\includegraphics[width=3cm]{etc/database log.jpg}
\end{minipage}

\vspace{0.5cm}

\noindent\rule{\textwidth}{1pt}

% \maketitle

\renewcommand{\abstractname}{طراحی پایگاه داده فروشگاه آنلاین}


\section{فاز سوم: ایجاد جدول واضافه کردن دیتا  - مهلت تحویل: ۱ تیر}


پس از نرمال‌سازی داده‌ها، نوبت به ایجاد جداول در پایگاه داده \lr{MySQL} می‌رسد. پس از ایجاد جداول، باید داده‌های خود را به پایگاه داده وارد کنید.

برای اضافه کردن داده‌ها می‌توانید از دو روش استفاده کنید:

\textbf{روش اول:} در این روش، داده‌ها به صورت دستی وارد پایگاه داده می‌شوند (با استفاده از دستور \texttt{\lr{INSERT}}).

\textbf{روش دوم (امتیازی):} می‌توانید داده‌های جدول خود را با استفاده از وب اسکرپینگ (\lr{Web Scraping}) و یا تولید داده‌های ساختگی اما منطقی، یا ترکیبی از هر دو روش به دست آورید و سپس آن‌ها را با استفاده از زبان پایتون به پایگاه داده خود اضافه کنید. (\href{https://github.com/parvvaresh/Online-Shop-database-design/blob/main/Phase3/insert_data.py}{کد پایتون برای اضافه کردن داده‌ها})

اگر از روش وب اسکرپینگ استفاده می‌کنید، می‌توانید از کتابخانه‌های \texttt{\lr{BeautifulSoup}} (\lr{bs4}) و \texttt{\lr{Requests}} استفاده کنید. استفاده از کدهای آماده موجود در \lr{GitHub} نیز بلامانع است.

شما باید یک فایل \lr{ZIP} شامل پایگاه داده خود و داده‌های موجود در آن، و در صورت استفاده از روش امتیازی، کدهای مربوطه را ارسال کنید.

در صورت هر گونه ابهام و سوال به این ایدی ها پیام دهید :‌

\href{https://t.me/Ali_Abdollahian_Noghabi}{پرورش}

\href{https://t.me/parvvaresh}{‫ﻋﺒﺪﺍﻟﻬﻴﺎﻥ‬}



\end{document}
